
\let\negmedspace\undefined
\let\negthickspace\undefined
\documentclass[article]{IEEEtran}
       \def\inputGnumericTable{}                                 %%
\usepackage{cite}
\usepackage{amsmath,amssymb,amsfonts,amsthm}
\usepackage{algorithmic}
\usepackage{graphicx}
\usepackage{textcomp}
\usepackage{xcolor}
\usepackage{txfonts}
\usepackage{listings}
\usepackage{enumitem}
\usepackage{mathtools}
\usepackage{gensymb}
\usepackage[breaklinks=true]{hyperref}
\usepackage{tkz-euclide} % loads  TikZ and tkz-base
\usepackage{listings}
\renewcommand{\theenumi}{\Alph{enumi}}
%
%\usepackage{setspace}
%\usepackage{gensymb}
%\doublespacing
%\singlespacing

%\usepackage{graphicx}
%\usepackage{amssymb}
%\usepackage{relsize}
%\usepackage[cmex10]{amsmath}
%\usepackage{amsthm}
%\interdisplaylinepenalty=2500
%\savesymbol{iint}
%\usepackage{txfonts}
%\restoresymbol{TXF}{iint}
%\usepackage{wasysym}
%\usepackage{amsthm}
%\usepackage{iithtlc}
%\usepackage{mathrsfs}
%\usepackage{txfonts}
%\usepackage{stfloats}
%\usepackage{bm}
%\usepackage{cite}
%\usepackage{cases}
%\usepackage{subfig}
%\usepackage{xtab}
%\usepackage{longtable}
%\usepackage{multirow}
%\usepackage{algorithm}
%\usepackage{algpseudocode}
%\usepackage{enumitem}
%\usepackage{mathtools}
%\usepackage{tikz}
%\usepackage{circuitikz}
%\usepackage{verbatim}
%\usepackage{tfrupee}
%\usepackage{stmaryrd}
%\usetkzobj{all}
    \usepackage{color}                                            %%
    \usepackage{array}                                            %%
    \usepackage{longtable}                                        %%
    \usepackage{calc}                                             %%
    \usepackage{multirow}                                         %%
    \usepackage{hhline}                                           %%
    \usepackage{ifthen}                                           %%
 %optionally (for landscape tables embedded in another document): %%
    \usepackage{lscape}     
%\usepackage{multicol}
%\usepackage{chngcntr}
%\usepackage{enumerate}

%\usepackage{wasysym}
%\documentclass[conference]{IEEEtran}
%\IEEEoverridecommandlockouts
% The preceding line is only needed to identify funding in the first footnote. If that is unneeded, please comment it out.

\newtheorem{theorem}{Theorem}[section]
\newtheorem{problem}{Problem}
\newtheorem{proposition}{Proposition}[section]
\newtheorem{lemma}{Lemma}[section]
\newtheorem{corollary}[theorem]{Corollary}
\newtheorem{example}{Example}[section]
\newtheorem{definition}[problem]{Definition}
%\newtheorem{thm}{Theorem}[section] 
%\newtheorem{defn}[thm]{Definition}
%\newtheorem{algorithm}{Algorithm}[section]
%\newtheorem{cor}{Corollary}
\newcommand{\BEQA}{\begin{eqnarray}}
\newcommand{\EEQA}{\end{eqnarray}}
\newcommand{\define}{\stackrel{\triangle}{=}}
\theoremstyle{remark}
\newtheorem{rem}{Remark}

\begin{document}
\providecommand{\pr}[1]{\ensuremath{\Pr\left(#1\right)}}
\providecommand{\prt}[2]{\ensuremath{p_{#1}^{\left(#2\right)} }}        % own macro for this question
\providecommand{\qfunc}[1]{\ensuremath{Q\left(#1\right)}}
\providecommand{\sbrak}[1]{\ensuremath{{}\left[#1\right]}}
\providecommand{\lsbrak}[1]{\ensuremath{{}\left[#1\right.}}
\providecommand{\rsbrak}[1]{\ensuremath{{}\left.#1\right]}}
\providecommand{\brak}[1]{\ensuremath{\left(#1\right)}}
\providecommand{\lbrak}[1]{\ensuremath{\left(#1\right.}}
\providecommand{\rbrak}[1]{\ensuremath{\left.#1\right)}}
\providecommand{\cbrak}[1]{\ensuremath{\left\{#1\right\}}}
\providecommand{\lcbrak}[1]{\ensuremath{\left\{#1\right.}}
\providecommand{\rcbrak}[1]{\ensuremath{\left.#1\right\}}}
\newcommand{\sgn}{\mathop{\mathrm{sgn}}}
\providecommand{\abs}[1]{\left\vert#1\right\vert}
\providecommand{\res}[1]{\Res\displaylimits_{#1}} 
\providecommand{\norm}[1]{\left\lVert#1\right\rVert}
%\providecommand{\norm}[1]{\lVert#1\rVert}
\providecommand{\mtx}[1]{\mathbf{#1}}
\providecommand{\mean}[1]{E\left[ #1 \right]}
\providecommand{\cond}[2]{#1\middle|#2}
\providecommand{\fourier}{\overset{\mathcal{F}}{ \rightleftharpoons}}
\newenvironment{amatrix}[1]{%
  \left(\begin{array}{@{}*{#1}{c}|c@{}}
}{%
  \end{array}\right)
}
%\providecommand{\hilbert}{\overset{\mathcal{H}}{ \rightleftharpoons}}
%\providecommand{\system}{\overset{\mathcal{H}}{ \longleftrightarrow}}
	%\newcommand{\solution}[2]{\textbf{Solution:}{#1}}
\newcommand{\solution}{\noindent \textbf{Solution: }}
\newcommand{\cosec}{\,\text{cosec}\,}
\providecommand{\dec}[2]{\ensuremath{\overset{#1}{\underset{#2}{\gtrless}}}}
\newcommand{\myvec}[1]{\ensuremath{\begin{pmatrix}#1\end{pmatrix}}}
\newcommand{\mydet}[1]{\ensuremath{\begin{vmatrix}#1\end{vmatrix}}}
\newcommand{\myaugvec}[2]{\ensuremath{\begin{amatrix}{#1}#2\end{amatrix}}}
\providecommand{\rank}{\text{rank}}
\providecommand{\pr}[1]{\ensuremath{\Pr\left(#1\right)}}
\providecommand{\qfunc}[1]{\ensuremath{Q\left(#1\right)}}
	\newcommand*{\permcomb}[4][0mu]{{{}^{#3}\mkern#1#2_{#4}}}
\newcommand*{\perm}[1][-3mu]{\permcomb[#1]{P}}
\newcommand*{\comb}[1][-1mu]{\permcomb[#1]{C}}
\providecommand{\qfunc}[1]{\ensuremath{Q\left(#1\right)}}
\providecommand{\gauss}[2]{\mathcal{N}\ensuremath{\left(#1,#2\right)}}
\providecommand{\diff}[2]{\ensuremath{\frac{d{#1}}{d{#2}}}}
\providecommand{\myceil}[1]{\left \lceil #1 \right \rceil }
\newcommand\figref{Fig.~\ref}
\newcommand\tabref{Table~\ref}
\newcommand{\sinc}{\,\text{sinc}\,}
\newcommand{\rect}{\,\text{rect}\,}
%%
%	%\newcommand{\solution}[2]{\textbf{Solution:}{#1}}
%\newcommand{\solution}{\noindent \textbf{Solution: }}
%\newcommand{\cosec}{\,\text{cosec}\,}
%\numberwithin{equation}{section}
%\numberwithin{equation}{subsection}
%\numberwithin{problem}{section}
%\numberwithin{definition}{section}
%\makeatletter
%\@addtoreset{figure}{problem}
%\makeatother

%\let\StandardTheFigure\thefigure
\let\vec\mathbf

\bibliographystyle{IEEEtran}
\title{
%	\logo{
Assignment
%	}
}
\author{ Karthikeya hanu prakash kanithi (EE22BTECH11026)}
\maketitle
\parindent0px
\vspace{3cm}
Question : Suppose that $(X, Y)$ has joint probability mass function
\begin{align}
P(X = 0, Y = 0) &= P(X = 1, Y = 1) = \theta, \\
P(X = 1, Y = 0) &= P(X = 0, Y = 1) = \frac{1}{2} - \theta.
\end{align}
where $0 \le \theta \le \frac{1}{2}$ is an unknown parameter. Consider testing $H_0 : \theta = \frac{1}{4}$ against $H_1 : \theta = \frac{1}{3}$; based on a random sample ${(X_1 , Y_1 ), (X_2 , Y_2 ), \ldots (X_n , Y_n )}$ from the above probability mass function. Let $M$ be the cardinality of the set $\{i: X_i = Y_i , 1 \le i\le n\}$. If $m$ is the observed value of $M$, then which one of the following statements is true?
\begin{enumerate}
\item The likelihood ratio test rejects $H_0$ if $m > c$ for some $c$.
\item The likelihood ratio test rejects $H_0$ if $m < c$ for some $c$.
\item The likelihood ratio test rejects $H_0$ if $c_1 < m < c_2$ for some $c_1$ and $c_2$.
\item The likelihood ratio test rejects $H_0$ if $m < c_1$ or $m > c_2$ for some $c_1$ and $c_2$.
\end{enumerate}
\solution 
Given that,
\begin{align}
	H_0 : \quad \theta = \theta_0 = \frac{1}{4},\\
	H_1 : \quad \theta = \theta_1 = \frac{1}{3}.
\end{align}
and the pmf is given by 
\begin{align}
	p_{XY}(0,0) &= p_{XY}(1,1) = \theta \\
	p_{XY}(0,1) &= p_{XY}(1,0) = \frac{1}{2} - \theta 
\end{align}
Then for the given random sample of data, 
\begin{align}
    \pr{X_i, Y_i} &= 
    \begin{cases}
        2\theta &  X_i=Y_i  \\
        1 - 2\theta & X_i\ne Y_i
    \end{cases} \\
\end{align}
Then the likelihood of the data under $H_0$ is given by: 
\begin{align}
    L(\theta_0 \mid data) &= \prod_{i=1}^{n} \pr{X_i, Y_i} \\
    &= \brak{2\theta_0}^m\brak{1 - 2\theta_0}^{n-m}\\
    &= \brak{\frac{1}{2}}^m\brak{\frac{1}{2}}^{n-m}
\end{align}
Then the likelihood of the data under $H_1$ is given by:
\begin{align}
    L(\theta_1 \mid data) &= \prod_{i=1}^{n} \pr{X_i, Y_i} \\
    &= \brak{2\theta_1}^m\brak{1 - 2\theta_1}^{n-m}\\
    &= \brak{\frac{2}{3}}^m\brak{\frac{1}{3}}^{n-m}
\end{align}
The likelyhood ratio will be 
\begin{align}
    \lambda(data) &= \frac{L(\theta_1 \mid x)}{L(\theta_0 \mid x)} \\
    &= \frac{\brak{\frac{2}{3}}^m\brak{\frac{1}{3}}^{n-m}}{\brak{\frac{1}{2}}^m\brak{\frac{1}{2}}^{n-m}} = \brak{2}^m\brak{\frac{2}{3}}^{n} \label{eq:st/42/1}
\end{align}
Let the critical value be denoted by $c_1$, then the likelihood ratio test rejects $H_0$ if
\begin{align}
    \implies  \lambda(data) &\overset{H_1}{\underset{H_0}{\gtrless}} c_1\\
\end{align}  
From \eqref{eq:st/42/1},
\begin{align}
    \implies  \brak{2}^m\brak{\frac{2}{3}}^{n} &\overset{H_1}{\underset{H_0}{\gtrless}} c_1\\
    \implies  \brak{2}^m &\overset{H_1}{\underset{H_0}{\gtrless}} c_1\brak{\frac{2}{3}}^{n}\\
    \implies  m &\overset{H_1}{\underset{H_0}{\gtrless}} \log_{2}\brak{c_1\brak{\frac{2}{3}}}^{n}\\
    \implies  m &\overset{H_1}{\underset{H_0}{\gtrless}} c \quad \exists \, c \in \mathbb{R} \label{eq:st/42/2}
\end{align}
$\therefore$ From \eqref{eq:st/42/2}, Option A is correct and Options B,C,D are incorrect
\end{document}
