
\let\negmedspace\undefined
\let\negthickspace\undefined
\documentclass[article]{IEEEtran}
       \def\inputGnumericTable{}                                 %%
\usepackage{cite}
\usepackage{amsmath,amssymb,amsfonts,amsthm}
\usepackage{algorithmic}
\usepackage{graphicx}
\usepackage{textcomp}
\usepackage{xcolor}
\usepackage{txfonts}
\usepackage{listings}
\usepackage{enumitem}
\usepackage{mathtools}
\usepackage{gensymb}
\usepackage[breaklinks=true]{hyperref}
\usepackage{tkz-euclide} % loads  TikZ and tkz-base
\usepackage{listings}
\renewcommand{\theenumi}{\Alph{enumi}}
%
%\usepackage{setspace}
%\usepackage{gensymb}
%\doublespacing
%\singlespacing

%\usepackage{graphicx}
%\usepackage{amssymb}
%\usepackage{relsize}
%\usepackage[cmex10]{amsmath}
%\usepackage{amsthm}
%\interdisplaylinepenalty=2500
%\savesymbol{iint}
%\usepackage{txfonts}
%\restoresymbol{TXF}{iint}
%\usepackage{wasysym}
%\usepackage{amsthm}
%\usepackage{iithtlc}
%\usepackage{mathrsfs}
%\usepackage{txfonts}
%\usepackage{stfloats}
%\usepackage{bm}
%\usepackage{cite}
%\usepackage{cases}
%\usepackage{subfig}
%\usepackage{xtab}
%\usepackage{longtable}
%\usepackage{multirow}
%\usepackage{algorithm}
%\usepackage{algpseudocode}
%\usepackage{enumitem}
%\usepackage{mathtools}
%\usepackage{tikz}
%\usepackage{circuitikz}
%\usepackage{verbatim}
%\usepackage{tfrupee}
%\usepackage{stmaryrd}
%\usetkzobj{all}
    \usepackage{color}                                            %%
    \usepackage{array}                                            %%
    \usepackage{longtable}                                        %%
    \usepackage{calc}                                             %%
    \usepackage{multirow}                                         %%
    \usepackage{hhline}                                           %%
    \usepackage{ifthen}                                           %%
 %optionally (for landscape tables embedded in another document): %%
    \usepackage{lscape}     
%\usepackage{multicol}
%\usepackage{chngcntr}
%\usepackage{enumerate}

%\usepackage{wasysym}
%\documentclass[conference]{IEEEtran}
%\IEEEoverridecommandlockouts
% The preceding line is only needed to identify funding in the first footnote. If that is unneeded, please comment it out.

\newtheorem{theorem}{Theorem}[section]
\newtheorem{problem}{Problem}
\newtheorem{proposition}{Proposition}[section]
\newtheorem{lemma}{Lemma}[section]
\newtheorem{corollary}[theorem]{Corollary}
\newtheorem{example}{Example}[section]
\newtheorem{definition}[problem]{Definition}
%\newtheorem{thm}{Theorem}[section] 
%\newtheorem{defn}[thm]{Definition}
%\newtheorem{algorithm}{Algorithm}[section]
%\newtheorem{cor}{Corollary}
\newcommand{\BEQA}{\begin{eqnarray}}
\newcommand{\EEQA}{\end{eqnarray}}
\newcommand{\define}{\stackrel{\triangle}{=}}
\theoremstyle{remark}
\newtheorem{rem}{Remark}

\begin{document}
\providecommand{\pr}[1]{\ensuremath{\Pr\left(#1\right)}}
\providecommand{\prt}[2]{\ensuremath{p_{#1}^{\left(#2\right)} }}        % own macro for this question
\providecommand{\qfunc}[1]{\ensuremath{Q\left(#1\right)}}
\providecommand{\sbrak}[1]{\ensuremath{{}\left[#1\right]}}
\providecommand{\lsbrak}[1]{\ensuremath{{}\left[#1\right.}}
\providecommand{\rsbrak}[1]{\ensuremath{{}\left.#1\right]}}
\providecommand{\brak}[1]{\ensuremath{\left(#1\right)}}
\providecommand{\lbrak}[1]{\ensuremath{\left(#1\right.}}
\providecommand{\rbrak}[1]{\ensuremath{\left.#1\right)}}
\providecommand{\cbrak}[1]{\ensuremath{\left\{#1\right\}}}
\providecommand{\lcbrak}[1]{\ensuremath{\left\{#1\right.}}
\providecommand{\rcbrak}[1]{\ensuremath{\left.#1\right\}}}
\newcommand{\sgn}{\mathop{\mathrm{sgn}}}
\providecommand{\abs}[1]{\left\vert#1\right\vert}
\providecommand{\res}[1]{\Res\displaylimits_{#1}} 
\providecommand{\norm}[1]{\left\lVert#1\right\rVert}
%\providecommand{\norm}[1]{\lVert#1\rVert}
\providecommand{\mtx}[1]{\mathbf{#1}}
\providecommand{\mean}[1]{E\left[ #1 \right]}
\providecommand{\cond}[2]{#1\middle|#2}
\providecommand{\fourier}{\overset{\mathcal{F}}{ \rightleftharpoons}}
\newenvironment{amatrix}[1]{%
  \left(\begin{array}{@{}*{#1}{c}|c@{}}
}{%
  \end{array}\right)
}
%\providecommand{\hilbert}{\overset{\mathcal{H}}{ \rightleftharpoons}}
%\providecommand{\system}{\overset{\mathcal{H}}{ \longleftrightarrow}}
	%\newcommand{\solution}[2]{\textbf{Solution:}{#1}}
\newcommand{\solution}{\noindent \textbf{Solution: }}
\newcommand{\cosec}{\,\text{cosec}\,}
\providecommand{\dec}[2]{\ensuremath{\overset{#1}{\underset{#2}{\gtrless}}}}
\newcommand{\myvec}[1]{\ensuremath{\begin{pmatrix}#1\end{pmatrix}}}
\newcommand{\mydet}[1]{\ensuremath{\begin{vmatrix}#1\end{vmatrix}}}
\newcommand{\myaugvec}[2]{\ensuremath{\begin{amatrix}{#1}#2\end{amatrix}}}
\providecommand{\rank}{\text{rank}}
\providecommand{\pr}[1]{\ensuremath{\Pr\left(#1\right)}}
\providecommand{\qfunc}[1]{\ensuremath{Q\left(#1\right)}}
	\newcommand*{\permcomb}[4][0mu]{{{}^{#3}\mkern#1#2_{#4}}}
\newcommand*{\perm}[1][-3mu]{\permcomb[#1]{P}}
\newcommand*{\comb}[1][-1mu]{\permcomb[#1]{C}}
\providecommand{\qfunc}[1]{\ensuremath{Q\left(#1\right)}}
\providecommand{\gauss}[2]{\mathcal{N}\ensuremath{\left(#1,#2\right)}}
\providecommand{\diff}[2]{\ensuremath{\frac{d{#1}}{d{#2}}}}
\providecommand{\myceil}[1]{\left \lceil #1 \right \rceil }
\newcommand\figref{Fig.~\ref}
\newcommand\tabref{Table~\ref}
\newcommand{\sinc}{\,\text{sinc}\,}
\newcommand{\rect}{\,\text{rect}\,}
%%
%	%\newcommand{\solution}[2]{\textbf{Solution:}{#1}}
%\newcommand{\solution}{\noindent \textbf{Solution: }}
%\newcommand{\cosec}{\,\text{cosec}\,}
%\numberwithin{equation}{section}
%\numberwithin{equation}{subsection}
%\numberwithin{problem}{section}
%\numberwithin{definition}{section}
%\makeatletter
%\@addtoreset{figure}{problem}
%\makeatother

%\let\StandardTheFigure\thefigure
\let\vec\mathbf

\bibliographystyle{IEEEtran}
\title{
%	\logo{
Assignment
%	}
}
\author{ Karthikeya hanu prakash kanithi (EE22BTECH11026)}
\maketitle
\parindent0px
\vspace{3cm}
Question : Let \{0.13, 0.12, 0.78, 0.51\} be a realization of a random sample of size 4 from a population with cumulative distribution function $F(.)$. Consider testing
\begin{align}
H_0 : F = F_0 \quad \text{against} \quad H_1 : F \ne F_0
\end{align}
where,
\begin{align}
    F_0(x) &= 
    \begin{cases}
        0 &  x<0  \\
        x & 0\le x<1 \\
        1 & x\ge 1
    \end{cases}
\end{align}
Let $D$ denote the Kolmogorov-Smirnov test statistic. If $P (D > 0.669) = 0.01$ under $H_0$ and
\begin{align}
    \psi &= 
    \begin{cases}
        1 &  \text{if } H_0 \text{ is accepted at level } 0.01 \\
        0 &  \text{otherwise} 
    \end{cases} 
\end{align}
then based on the given data, the observed value of $D + \psi$ (rounded off to two decimal places) equals
\\\solution 
Its given that random sample is of size 4, So 
\begin{align}
	n=4 \label{eq:st/65/1}
\end{align}
The cdf of the random sample is given as 
\begin{align}
    F_X(x) &= 
    \begin{cases}
        0 &  x<0  \\
        x & 0\le x<1 \\
        1 & x\ge 1
    \end{cases} \label{eq:st/65/2}
\end{align}
The empirical distribution function(edf) $F_n$ for n independent and identically distributed (i.i.d.) ordered observations $X_i$ is defined as
\begin{align}
	F_n(x) = \frac{\text{no of (elements in the sample} \le x)}{n} = \frac{1}{n} \sum_{i=1}^{n} 1_{X_i \le x}  \label{eq:st/65/3}
\end{align}
where $1_A$ is the indicator of event A.\\
From \eqref{eq:st/65/1}, \eqref{eq:st/65/2} and \eqref{eq:st/65/3}, the edf for the given data will be
\begin{align}
  	F_n(0.13) = \frac{1}{4} \sum_{i=1}^{n} 1_{X_i \le 0.13} = \frac{1}{2}	\\
  	F_n(0.12) = \frac{1}{4} \sum_{i=1}^{n} 1_{X_i \le 0.12} = \frac{1}{4}\\
  	F_n(0.78) = \frac{1}{4} \sum_{i=1}^{n} 1_{X_i \le 0.78} = 1\\
  	F_n(0.51) = \frac{1}{4} \sum_{i=1}^{n} 1_{X_i \le 0.51} = \frac{3}{4}
\end{align}
The Kolmogorov–Smirnov statistic for a given cdf $F_X(x)$ is
\begin{align}
  	D_n = \sup\abs{F_n(x) - F_X(x)} 
\end{align}
The difference between cdf and edf for the given data will be (i.e., $\forall x \in \{0.13, 0.12, 0.78, 0.51\} $)
\begin{align}
  	F_n(0.13) - F_X(0.13) = 0.37\\
  	F_n(0.12) - F_X(0.12) = 0.25\\
  	F_n(0.78) - F_X(0.78) = 0.22\\
  	F_n(0.51) - F_X(0.51) = 0.24
\end{align}
Then 
\begin{align}
  	D_n = \sup(0.37,0.25,0.22,0.24) = 0.37 \label{eq:st/65/4}
\end{align}
Given that,
\begin{align}
  	P (D > 0.669) = 0.01
\end{align}
Then
\begin{align}
H_0 =
\begin{cases}
\text{accepted at level } 0.01 & \text{if } D_n \le 0.669 \\
\text{rejected at level } 0.01 & \text{if } D_n > 0.669
\end{cases}\label{eq:st/65/5}
\end{align}
From \eqref{eq:st/65/4} and \eqref{eq:st/65/5}; We can say that $H_0$ is accepted at level 0.01 and 
\begin{align}
  	\psi = 1 
\end{align}
$\therefore$ the value will be 
\begin{align}
  	\psi + D_n = 1 + 0.37 = 1.37
\end{align}
\end{document}
