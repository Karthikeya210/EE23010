\documentclass{article}

\usepackage{graphicx}
\usepackage{geometry}
\usepackage{listings}
\usepackage{xcolor}

\geometry{a4paper, margin=1in}

\definecolor{myblack}{RGB}{0, 0, 0} % Black color
\definecolor{mycyan}{RGB}{0, 255, 255} % Cyan color
\definecolor{mygold}{RGB}{218, 165, 32} % Gold color

\pagecolor{myblack}
\color{mycyan}

\title{\textcolor{mygold}{Probability Course Grading Report}}
\author{\textcolor{mygold}{Karthikeya Hanu Prakash Kanithi}}

\begin{document}

\maketitle

\section{\textcolor{mygold}{Introduction}}

This report outlines the grading process for the course 'EE23010: Probability and Random Processes' in Fall 2023, offered by Dr. G V V Sharma using k-means clustering. The goal is to assign grades based on student performance, as determined by clustering their marks.

\section{\textcolor{mygold}{Data and Preprocessing}}

The marks data was obtained from the course assessments and stored in an Excel file. The Python script used for clustering reads the data, standardizes it, and applies k-means clustering with seven clusters.

\section{\textcolor{mygold}{K-Means Clustering}}

\subsection{\textcolor{mygold}{Overview}}

K-means clustering is an unsupervised machine learning algorithm used for partitioning a dataset into K distinct, non-overlapping subsets (clusters). The algorithm works iteratively to assign each data point to one of K clusters based on its features.

\subsection{\textcolor{mygold}{Algorithm Steps}}

\begin{enumerate}
    \item \textbf{Initialization:} Choose K initial cluster centroids randomly.
    \item \textbf{Assignment:} Assign each data point to the nearest centroid, forming K clusters.
    \item \textbf{Update:} Recalculate the centroids based on the mean of data points in each cluster.
    \item \textbf{Repeat:} Repeat steps 2-3 until convergence or a maximum number of iterations.
\end{enumerate}

\section{\textcolor{mygold}{Grading Based on Clusters}}

The resulting clusters are mapped to grades as follows: ( CLusters are randomly generated)

\begin{itemize}
    \item \textbf{Cluster 0:} B-
    \item \textbf{Cluster 1:} A-
    \item \textbf{Cluster 2:} D
    \item \textbf{Cluster 3:} B
    \item \textbf{Cluster 4:} A
    \item \textbf{Cluster 5:} C-
    \item \textbf{Cluster 6:} C
\end{itemize}

Each student is assigned a grade based on the cluster to which their performance belongs.

\section{\textcolor{mygold}{Conclusion}}

This grading process provides a systematic way to categorize student performance in the Probability course. The use of k-means clustering helps in objectively determining grade boundaries.

\end{document}

