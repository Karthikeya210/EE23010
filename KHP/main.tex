\documentclass{beamer}
\usepackage{amsmath,amssymb,amsfonts,amsthm}
\usepackage{algorithmic}
\usepackage{graphicx}
\usepackage{textcomp}
\usepackage{xcolor}
\usepackage{txfonts}
\usepackage{listings}
\usepackage{mathtools}
\usepackage{gensymb}
\usepackage{hyperref}
\usepackage{tkz-euclide} % loads  TikZ and tkz-base
\usepackage{listings}
    \usepackage{color}                                            %%
    \usepackage{array}                                            %%
    \usepackage{longtable}                                        %%
    \usepackage{calc}                                             %%
    \usepackage{multirow}                                         %%
    \usepackage{hhline}                                           %%
    \usepackage{ifthen}    
    \usepackage{lscape}
\usetheme{Frankfurt}
\providecommand{\abs}[1]{\left\vert#1\right\vert}
\title{Probability with Coding}
\author{Karthikeya Hanu Prakash Kanithi}
\institute{IIT Hyd}
\date{\today}

\begin{document}
\lstset{
    language=Python,   % Set the programming language for syntax highlighting
    basicstyle=\ttfamily, % Set the font style for the code
    keywordstyle=\color{blue}, % Customize keywords
    commentstyle=\color{green}, % Customize comments
    stringstyle=\color{red},   % Customize strings
    numbers=left,      % Display line numbers
    numberstyle=\tiny, % Set the style for line numbers
    breaklines=true,   % Automatically wrap long lines
}
\lstset{
    language=C,               % Set the programming language
    basicstyle=\ttfamily,     % Font style for the code
    keywordstyle=\color{blue}, % Keyword style
    commentstyle=\color{green},% Comment style
    stringstyle=\color{red},   % String style
    numbers=left,             % Display line numbers
    numberstyle=\tiny,        % Style for line numbers
    breaklines=true,           % Automatically wrap long lines
    frame=single,             % Add a frame around the code
    showstringspaces=false,    % Don't show spaces within strings
    tabsize=4,                % Set tab size to 4 spaces
    morekeywords={printf, scanf, int, main, if, else, while}, % Additional keywords
    extendedchars=true,       % Allow extended characters like underscores
    literate={~} {$\sim$}{1}, % Replace ~ with a tilde symbol
    backgroundcolor=\color{gray!10}, % Background color for the code
    escapeinside={(*@}{@*)},  % Define an escape sequence for LaTeX code within comments
}
\providecommand{\pr}[1]{\ensuremath{\Pr\left(#1\right)}}
\providecommand{\prt}[2]{\ensuremath{p_{#1}^{\left(#2\right)} }}        % own macro for this question
\providecommand{\qfunc}[1]{\ensuremath{Q\left(#1\right)}}
\providecommand{\sbrak}[1]{\ensuremath{{}\left[#1\right]}}
\providecommand{\lsbrak}[1]{\ensuremath{{}\left[#1\right.}}
\providecommand{\rsbrak}[1]{\ensuremath{{}\left.#1\right]}}
\providecommand{\brak}[1]{\ensuremath{\left(#1\right)}}
\providecommand{\lbrak}[1]{\ensuremath{\left(#1\right.}}
\providecommand{\rbrak}[1]{\ensuremath{\left.#1\right)}}
\providecommand{\cbrak}[1]{\ensuremath{\left\{#1\right\}}}
\providecommand{\lcbrak}[1]{\ensuremath{\left\{#1\right.}}
\providecommand{\rcbrak}[1]{\ensuremath{\left.#1\right\}}}
\newcommand{\sgn}{\mathop{\mathrm{sgn}}}
\providecommand{\abs}[1]{\left\vert#1\right\vert}
\providecommand{\res}[1]{\Res\displaylimits_{#1}} 
\providecommand{\norm}[1]{\left\lVert#1\right\rVert}
%\providecommand{\norm}[1]{\lVert#1\rVert}
\providecommand{\mtx}[1]{\mathbf{#1}}
\providecommand{\mean}[1]{E\left[ #1 \right]}
\providecommand{\cond}[2]{#1\middle|#2}
\providecommand{\fourier}{\overset{\mathcal{F}}{ \rightleftharpoons}}
\newenvironment{amatrix}[1]{%
  \left(\begin{array}{@{}*{#1}{c}|c@{}}
}{%
  \end{array}\right)
}
\newcommand{\cosec}{\,\text{cosec}\,}
\providecommand{\dec}[2]{\ensuremath{\overset{#1}{\underset{#2}{\gtrless}}}}
\newcommand{\myvec}[1]{\ensuremath{\begin{pmatrix}#1\end{pmatrix}}}
\newcommand{\mydet}[1]{\ensuremath{\begin{vmatrix}#1\end{vmatrix}}}
\newcommand{\myaugvec}[2]{\ensuremath{\begin{amatrix}{#1}#2\end{amatrix}}}
\providecommand{\rank}{\text{rank}}
\providecommand{\pr}[1]{\ensuremath{\Pr\left(#1\right)}}
\providecommand{\qfunc}[1]{\ensuremath{Q\left(#1\right)}}
	\newcommand*{\permcomb}[4][0mu]{{{}^{#3}\mkern#1#2_{#4}}}
\newcommand*{\perm}[1][-3mu]{\permcomb[#1]{P}}
\newcommand*{\comb}[1][-1mu]{\permcomb[#1]{C}}
\providecommand{\qfunc}[1]{\ensuremath{Q\left(#1\right)}}
\providecommand{\gauss}[2]{\mathcal{N}\ensuremath{\left(#1,#2\right)}}
\providecommand{\diff}[2]{\ensuremath{\frac{d{#1}}{d{#2}}}}
\providecommand{\myceil}[1]{\left \lceil #1 \right \rceil }
\newcommand\figref{Fig.~\ref}
\newcommand\tabref{Table~\ref}
\newcommand{\sinc}{\,\text{sinc}\,}
\newcommand{\rect}{\,\text{rect}\,}

\begin{frame}
  \titlepage
\end{frame}

\begin{frame}{Objectives}
  \begin{itemize}
  \item Generate standard normal random variable using C
  \item Solve our problem statement 
  \end{itemize}
\end{frame}

\begin{frame}{Python vs C}
  \begin{itemize}
  \item Python already has inbuilt libraries through which we can generate standard normal random variables. But In C we have to generate them using random numbers...
  \item Here is where we will learn about Box-muller transforms 
  \end{itemize}
\end{frame}

\begin{frame}{Box-Muller Transform}
  \begin{itemize}
  \item Suppose U1 and U2 are independent samples chosen from the uniform distribution on the unit interval (0, 1). Let
  \begin{align}
  	Z_0 = R \cos(\Theta) =\sqrt{-2 \ln U_1} \cos(2 \pi U_2)\,
  \end{align}
  \item  and
  \begin{align}
  	Z_1 = R \sin(\Theta) =\sqrt{-2 \ln U_1} \cos(2 \pi U_2)\, 
  \end{align}
  \item Then $Z_0$ and $Z_1$ are independent random variables with a standard normal distribution.
  \item So, now we will generate $Z_0$ using C Code as given below 
  \end{itemize}
\end{frame}

\begin{frame}[allowframebreaks]{C-code}
  \lstinputlisting[language=C]{/home/sayyam/KHP/codes/rand.c}
  % Include your LaTeX equations and calculations here...
\end{frame}

\begin{frame}{Python-code}
  \lstinputlisting[language=Python]{/home/sayyam/KHP/codes/rand.py}
  % Include your LaTeX equations and calculations here...
\end{frame}

\begin{frame}{Python-code}
\begin{figure}
  \centering
  \includegraphics[width=\columnwidth]{/home/sayyam/KHP/figs/figure1.png}  % Replace 'image_filename' with your image file name
  \caption{Histogram plot of density of $Z_0$}
  \label{fig:your_label}
\end{figure}
\end{frame}

\begin{frame}{Problem Statement}
  Let $\phi(.)$ denote the cumulative distribution function of a standard normal
random variable. If the random variable $X$ has the cumulative distribution
function 
\begin{align}
	F(x)&= 
    \begin{cases}
        \phi(x), &  x < -1 \\
        \phi(x+1) , &  x \ge -1
    \end{cases} \label{eq:15st/2023}
\end{align}
then which one of the following statements is true?
\begin{enumerate}
\item $P(X \leq -1) = \frac{1}{2}$
\item $P(X = -1) = \frac{1}{2}$
\item $P(X < -1) = \frac{1}{2}$
\item $P(X \leq 0) = \frac{1}{2}$
\end{enumerate}
\end{frame}

\begin{frame}[allowframebreaks]{C-code}
  \lstinputlisting[language=C]{/home/sayyam/KHP/codes1/rand.c}
  % Include your LaTeX equations and calculations here...
\end{frame}

\begin{frame}{Python-code}
  \lstinputlisting[language=Python]{/home/sayyam/KHP/codes1/rand.py}
  % Include your LaTeX equations and calculations here...
\end{frame}

\begin{frame}{Python-code}
\begin{figure}
  \centering
  \includegraphics[width=\columnwidth]{/home/sayyam/KHP/figs/figure2.png}  % Replace 'image_filename' with your image file name
  \caption{Histogram plot of density of $Z_0$}
  \label{fig:your_label}
\end{figure}
\end{frame}
\end{document}

